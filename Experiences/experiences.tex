\documentclass[12pt, a4paper, french]{article}

%--------PACKAGES--------%
\usepackage[utf8]{inputenc}
\usepackage[T1]{fontenc}
\usepackage{lmodern}
\usepackage{geometry}
\usepackage{amssymb}
\usepackage{mathtools}
\usepackage[squaren, cdot, derived]{SIunits} % \unit{<nb>}{<unite>}
\usepackage{graphicx}
\usepackage{enumitem}
\usepackage[f]{esvect}
\usepackage{xcolor}
\usepackage[french]{babel}

\title{Expériences physiques}
\author{\textsc{Andrieux} Martin\\
	\textsc{Maillet} Nathan}


\begin{document}
	\maketitle
	% Premier pb : comment justifier que le comportement des billes (et/ou des fluides) est analogue à celui des individus (ou pas) ?
	% Risque catalogue dans la présentation de plusieurs situations
	% Plusieurs expérience montrant la simple existance d'un phénomène (ex : écoulement de sable) puis vérification de cette même existance dans la simulation, risque d'être "catalogue" ?
	% Dans le but de valider ou d'invalider la simulation sur des points précis
	% Recevable ? Bien vu (par le jury) ? Utile ?
	\section{cadre des expériences}
	\begin{itemize}[label =$\bullet$]
		\item Limite des expériences
		\begin{itemize}[label =$\circ$]
			\item Liens entre le comportement de la bille et celui des individus
			\item Abscence de collisions pour les fluides
			\item Expérience humaine envisageable ?
		\end{itemize}
	\end{itemize}
	
	
	\section{Phénomènes à observer et résultats attendus}
	\begin{itemize}[label =$\bullet$]
		\item Arcs à la sortie : les entitées ne peuvent plus sortir
		\item Utilité des piliers : casse les arcs, repousse leur création ?
		\item Influence de la taille des sorties (temps de sortie $\propto$ taille de la sortie ?) 
		\item Contournements de piliers et de peronnes : V comme avec la simulation ?
		\item (Influence de la majorité (attraction))
	\end{itemize}
	
	
	
	\section{Protocoles d'expériences}
	\subsection{Expérience}
	\begin{itemize}[label =$\bullet$]
		\item Acheter des billes
		\item Faire un plateau avec du carton
	\end{itemize}
\end{document}